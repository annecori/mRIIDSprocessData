\documentclass[11pt,]{article}
\usepackage[]{mathpazo}
\usepackage{amssymb,amsmath}
\usepackage{subcaption}
\usepackage{ifxetex,ifluatex}
\usepackage{fixltx2e} % provides \textsubscript
\ifnum 0\ifxetex 1\fi\ifluatex 1\fi=0 % if pdftex
  \usepackage[T1]{fontenc}
  \usepackage[utf8]{inputenc}
\else % if luatex or xelatex
  \ifxetex
    \usepackage{mathspec}
  \else
    \usepackage{fontspec}
  \fi
  \defaultfontfeatures{Ligatures=TeX,Scale=MatchLowercase}
\fi
% use upquote if available, for straight quotes in verbatim environments
\IfFileExists{upquote.sty}{\usepackage{upquote}}{}
% use microtype if available
\IfFileExists{microtype.sty}{%
\usepackage{microtype}
\UseMicrotypeSet[protrusion]{basicmath} % disable protrusion for tt fonts
}{}
\usepackage[margin=1in]{geometry}
\usepackage{hyperref}
\hypersetup{unicode=true,
            pdftitle={Milestone 6},
            pdfborder={0 0 0},
            breaklinks=true}
\urlstyle{same}  % don't use monospace font for urls
\usepackage{natbib}
\bibliographystyle{plainnat}
\usepackage{graphicx,grffile}
\makeatletter
\def\maxwidth{\ifdim\Gin@nat@width>\linewidth\linewidth\else\Gin@nat@width\fi}
\def\maxheight{\ifdim\Gin@nat@height>\textheight\textheight\else\Gin@nat@height\fi}
\makeatother
% Scale images if necessary, so that they will not overflow the page
% margins by default, and it is still possible to overwrite the defaults
% using explicit options in \includegraphics[width, height, ...]{}
\setkeys{Gin}{width=\maxwidth,height=\maxheight,keepaspectratio}
\IfFileExists{parskip.sty}{%
\usepackage{parskip}
}{% else
\setlength{\parindent}{0pt}
\setlength{\parskip}{6pt plus 2pt minus 1pt}
}
\setlength{\emergencystretch}{3em}  % prevent overfull lines
\providecommand{\tightlist}{%
  \setlength{\itemsep}{0pt}\setlength{\parskip}{0pt}}
\setcounter{secnumdepth}{5}
% Redefines (sub)paragraphs to behave more like sections
\ifx\paragraph\undefined\else
\let\oldparagraph\paragraph
\renewcommand{\paragraph}[1]{\oldparagraph{#1}\mbox{}}
\fi
\ifx\subparagraph\undefined\else
\let\oldsubparagraph\subparagraph
\renewcommand{\subparagraph}[1]{\oldsubparagraph{#1}\mbox{}}
\fi

%%% Use protect on footnotes to avoid problems with footnotes in titles
\let\rmarkdownfootnote\footnote%
\def\footnote{\protect\rmarkdownfootnote}

%%% Change title format to be more compact
\usepackage{titling}

% Create subtitle command for use in maketitle
\newcommand{\subtitle}[1]{
  \posttitle{
    \begin{center}\large#1\end{center}
    }
}

\setlength{\droptitle}{-2em}
  \title{Mapping the Risk of International Infectious Disease Spread
    (MRIIDS)}
  \subtitle{A project funded through USAID’s ``Combating Zika and
    Future Threats: A Grand Challenge for Development'' program \\
    Milestone 6: Increased complexity of the simple model to include data from Milestone 5}
  \pretitle{\vspace{\droptitle}\centering\huge}
  \posttitle{\par}
  \author{}
  \preauthor{}\postauthor{}
  \predate{\centering\large\emph}
  \postdate{\par}
  \date{27 November, 2017}


\usepackage[section]{placeins}

\begin{document}
\maketitle
\tableofcontents

\newpage
\section{Milestone Description}\label{milestone-description}

Increased complexity of the simple model to include data from Milestone
5. Automated testing and validating procedures implemented where
possible. Explore procedures for multiple models comparison and accounting for
uncertainty.

\section{General Approach}

In Milestone 4, we presented a simple transmission model that made use of
historical case counts (data stream 1), information about the
transmissibility of the pathogen (data stream 2) and geographical
characterization (data stream 3) to predict future risk. 
In this approach, the geographical characterization was not fully integrated in the model:
first transmissibility was estimated from case count data, then future incidence was predicted, and 
finally, the predicted incidence was distributed geographically according to the spatial distribution 
and population density of each geographical unit.

To achieve the goals outlined in Milestone 6, we built on the model presented in Milestone 4 
(ML4) to integrate geographical information into our inference procedure.
Having achieved this, other characteristics defined at the geographical
scale of reference such as the health care capacity of a location, can be 
added to the model.

In essence, by integrating the spatial information into the inference
and prediction phases, we have developed a complex model that has the potential to
account for the multiple data streams initially described.

Therefore, our new complex model has the ability to:
\begin{itemize}
\item estimate model parameters: including parameters linked to geographical spread (or
potentially health care capacity,
\item predict the regional/international spread of Ebola, relying on those parameters' estimates. 
\end{itemize}

We have also established the procedure for the validation process
using historical data and are exploring various possibilities for multi-model comparison. 

For the model validation, we rely on both Promed/HealthMap incidence data, which are 
at the national scale, and WHO reported data which are at the district level. The advantage 
of using the more spatially refined WHO data is that it allows increased statistical power to 
infer the spatial parameters of the model (i.e. more movements occur at finer spatial scale, 
therefore the `signature' of movement in incidence data is more identifiable at finer scale).
This exercise could be viewed as:
\begin{itemize}
\item demonstrating the flexible nature of our framework, i.e. 
the model is designed to be flexible in term of the choice of spatial scale, and
\item a proof of concept to argue that reporting spatially refined incidence count can improve our ability to
predict spatial spread.
\end{itemize}

\section{Presentation of the model}\label{sec:model}

The number of cases at a location \(j\) at time \(t\) is given by the equation
\[
  I_{j, t} \sim Pois\left( \sum_{i = 1}^{n} {\left( p_{i \rightarrow j}
  R_{t, i} \sum_{s = 1}^{t}{I_{i, t - s} w_{s}}\right)} \right),
\]

where \(R_{t, i}\) is the reproduction number at location \(i\) at time
\(t\) and \(p_{i \rightarrow j}\) is the probability of moving from
location \(i\) to location \(j\). The quantity $R_{t, i}$ is the
reproduction number at time $t$ at location $i$. $R_{t, i}$ is
affected by a number of other factors e.g., the intrinsic
transmissibility of a pathogen, the health care capacity at location
$i$ etc. Its dependence on these factors is formalized as
\[ R_{t, i} := f(haq_i, R_0, t),\]
where $haq_i$ is an index/score quantifying the health care capacity at location 
$i$, $f$ denotes a function, $R_0$ is the basic reproduction number (data stream 2) and $t$ is time..

The probability of moving between locations is derived from the relative flow of populations between
locations. This latter quantity is estimated using a population flow
model such as a gravity model. Under a gravity model, the flow of individuals from area \(i\) to area \(j\),
\(\phi_{i \rightarrow j}\), is proportional to the product of the
populations of the two areas, \(N_i\) and \(N_j\) and inversely
proportional to the distance between them \(d_{i, j}\), all quantities
are raised to some power.
\[
  \phi_{i \rightarrow j} :=  \frac{N_i^{\alpha}N_j^{\beta}}{d_{i, j}^{\gamma}}.
\]

In practice, \( \alpha \) and \( \beta \) are assumed to be $1$. The
exponent \( \gamma \) modulates the effects distance on the flow of
populations. A large value of \( \gamma \) indicates that the
distances traveled by populations tend to be short.

The relative risk of spread at a location \(j\) from a location \(i\)
is thus the population flow into location \(j\) from location \(i\).

\[
  r_{i \rightarrow j}^{spread} = \frac{\phi_{i \rightarrow
  j}}{\sum_{x}{\phi_{i \rightarrow
  j}}}.
\]

The probability of movement from location \(i\) to location \(j\) is given by
\[  p_{i \rightarrow j} = (1 - p_{stay}^i) r_{i \rightarrow j}^{spread},\]

where \(p_{stay}^i\) is the probability of staying at location
\(i\). As the above equation indicates, by varying $p_{stay}^i$, we
can capture the dynamics of population flow across spatial units. For
instance, if \(p_{stay}^i\) is large, then the flow out of location
\(i\) would be small. Thus, if this parameter is geographically
heterogeneous, we obtain imbalanced flow of population (i.e. a source-sink dynamics). 

\subsection{Statistical inference of model parameters}

The parameters of the full model as presented in Section~\ref{sec:model} are: 
\begin{itemize}
\item $R_{t, i}$, the reproduction at time $t$,
\item $p_{stay}$, the probability of staying in location $i$, and 
\item $\gamma$, the exponent of the distance in the gravity model. 
\end{itemize}


The parameters can be estimated using maximum likelihood
estimation or estimating the posterior distribution of the parameters using
MCMC. Let the observed incidence time series at locations \(1\)
through \(n\) and time \(1, 2 \dots t\) be
\[
I = \begin{bmatrix}
    o_{1,1}       & o_{1,2}  & \dots & o_{1,n} \\
    o_{2,1}       & o_{2,2}  & \dots & o_{2,n} \\
    \hdotsfor{4} \\
    o_{t,1}       & o_{t,2}  & \dots & o_{t,n}
\end{bmatrix}
\]
where \(o_{i, j}\) is the observed incidence at time \(i\) at location
\(j\).
Then the likelihood of the model parameters given the
observations is proportional to the probability of the data given
model parameters.  The probability of $o_{j, t}$ given
the model parameters is:
\[ P(o_{j, t} \mid p_{stay}^i, \gamma, R_{i, t}) = e^{-\lambda_{j, t}}
  \frac{o_{j, t}^{\lambda_{j, t}}}{\lambda_{j, t} !}, \]
where $\lambda_{j, t}$ is given by
\[
  \lambda_{j, t} = \sum_{i = 1}^n{\left(p_{i \rightarrow j}R_{i, t} \sum_{s
        = 1}^t{I_{i, s}w_{t - s}} \right)}.\]

Thus assuming that each observation is independent, the likelihood of the parameters is proportional to
\[
\mathcal{L} = P(\{o_{j, t}\} \mid \{p_{stay}^i\}, \gamma, \{R_{i, t}\}) = 
 \prod_{t = 1}^{t}{e^{-\lambda_{i, t}} \frac{o_{i, t}^{\lambda_{i, t}}}{\lambda_{i, t} !}}.
\]
In practice, we estimate $R_{t, i}$ as an average over the past 2 or 3 week with sliding time windows.

Given this likelihood, we can write the joint posterior distribution of the parameter given the observed data as:
\[
P( \{p_{stay}^i\}, \gamma, \{R_{i, t}\} \mid \{o_{j, t}\}) \propto  
  \mathcal{L} \times P(\{R_{i,0}\}) P(\{p_{stay}^i\}) P(\gamma).
\]

Here, $P(\{R_{i,0}\})$ represents the prior distribution of the basic reproduction number. This prior distribution is 
influenced by data-stream 2 and the health capacity of the location $i$, as described above.

The other prior distributions,  $P(\{p_{stay}^i\})$ and $P(\gamma)$, could in principle reflect the influence of 
additional data sources such as prior information derived from flight data. 

\subsection{Multi-model Comparison}

Given the most general model formulation outlined above, multiple models could be formulated that can be 
viewed as simplification of the original model. For instance assuming  $\{R_{i, t}\}$ to be constant across geographical units, 
or assuming unity for the parameter $zgamma$ of the gravity.

Variants of the model would have distinct number of parameters, for instance assuming $\gamma =1$ would reduce the number
of parameters to be estimated by 1. 

Relying on data-driven and evidence based approaches, we seek to formulate the 
simplest model that account for patterns observed in the data. In such simplest model, all layers of complexity must be justified, i.e.
it is justified to estimate a parameter $\gamma$ (and not assume unity) if and only if the fit of our model to the observed 
data is significantly improved (in a statistical sense).

Relying on our MCMC estimation of the joint posterior, we can evaluate the goodness of fit of any models by calculating
the Deviance Information Criterion (DIC). Such DIC is a well established measure of goodness of fit, and commonly 
used for model selection. it promotes models that can best reproduce observed patterns while penalizing
for increased models complexity (i.e. increased number of parameters).

Using such `model selection' approach, each model is evaluated in turn, DICs are obtained and compared, and
we can rigorously selected the best model to produce the final predictions.

Alternatively, after evaluating each model, we can produce 'model averaged' predictions. Under this approach, the predictions
of each model are weighted according to the statistical of the particular model. Such approach has the advantage of accounting
for structural uncertainties, i.e. the very structure of the underlying model is treated as an unknown state.

The 'model averaged' predictions are finally obtained based on the 4 following steps:
\begin{itemize}
\item evaluate the goodness of fit (e.g. DIC) for each model,
\item calculate the difference ($\Delta_k$) between the DIC value of the best model and the DIC value for each of the other models,
\item compute the relative weight for each model as $w_k = \frac{exp(\frac{1}{2}\Delta_k)}{\sum_{m = 1}^{M}{exp(\frac{1}{2}\Delta_m)}}$, with $M$ the total number of models evaluated,
\item produce predictions by sequentially 1) randomly selecting a particular model according to its weight, 2) producing predictions based on the parameters' joint posterior distributions associated with this particular model.
\end{itemize}

We are currently in the process of implementing those approaches (simple model selection and multi-model averaging).
Once implementation is finalized, we will validate and contrast both approach, recognizing that while selecting a model 
is easier to implement, multi-model averaging should better
accounts for both parameters and model uncertainty, i.e. parametric and structural uncertainties.

%%%%%\subsection{Predicting Future Incidence Pattern and Geographical Spread}

\section{Implementation Details}
\subsection{Software Package mRIIDS}

The general approach outlined above relies on several data streams, an
inference framework, and a framework for projection. The code
developed as part of the project is available as an open
source R package that provide functions for pre-processing and
collating the various data streams as well as plug the data into
modules that will do the inference and the projection. 

The software
package will eventually be published on the R packages repository
(CRAN). At the moment, it is available on GitHub (github.com/annecori/mRIIDSprocessData).

\begin{figure}
   \centering
  \includegraphics[width=8cm, height = 6cm]{ms6-figures/github-screenshot}
  \label{fig:github}
  \caption{The software being developed for the project is available
    on GitHub.}
\end{figure}


The package will include extensive documentation in the form of
user-friendly help files and vignettes.


\begin{figure}
  \centering
  \includegraphics[width=8cm, height = 6cm]{ms6-figures/Helpfile-screenshot}
  \caption{An example of the documentation for the R package mRIIDS}
  \label{fig:helpfile}
\end{figure}

\subsection{Collating data for each data stream}
\begin{center}
\begin{figure}
\includegraphics[]{ms6-figures/workflow}
\caption{Workflow for the processing of the various data streams.}
\label{fig:workflow}
\end{figure}
\end{center}

Figure~\ref{fig:workflow} summarizes the steps involved in collating
the different data streams and in going from raw data to predictions. In Milestone 6, a step was added to the data pre-processing workflow
to remove outliers from data. The removal of outliers was done using Chebyshev Inequality
with sample mean (see~\cite{saw1984chebyshev}). Figure~\ref{fig:wf_example} illustrates
the results of each step in the pre-processing steps in the workflow.

\begin{figure}
  \centering
  \includegraphics{ms6-figures/liberia-preprocessing}
  \caption{Illustration of the pre-processing steps on HealthMap incidence
    data for Liberia.}
  \label{fig:wf_example}
\end{figure}

\subsection{Model training and validation using data from
WHO}\label{model-training-and-validation-using-data-from-who}

In the current iteration, the model was trained and validated the data
on cases officially reported to the WHO during the 2013--2016 Ebola
outbreak in Guinea, Liberia and Sierra Leone. This dataset was cleaned
and published in \citep{garske20160308} and it is this cleaned version
of the data that were used in this work. This dataset consists of
incidence reports at ADM2 level. Thus in using it, we were able to validate the model at a finer spatial resolution than available
with HealthMap/ProMed data. We refer to this dataset as WHO data
throughout the rest of this document.

\subsection{Incidence trends from different data sources}
We aggregated the WHO data to national level to compare the incidence
trends derived from the three different data sources (WHO, HealthMap
and ProMed). As can be seen in Figure~\ref{fig:incid_comp}, the
three data sources correlate well in the incidence time series.

\begin{figure}
    \centering
        \includegraphics[width=\textwidth]{ms6-figures/who_vs_hm_vs_pm-incid}
        \caption{Comparison of incidence data from WHO, HealthMap and ProMed.}
        \label{fig:incid_comp}
  \end{figure}

\subsection{Inference of parameters}

The parameters of the full model detailed in Section~\ref{sec:model} are
$\alpha$, $\beta$, $\gamma$, $p_{stay}$ and $R_{i, t}$. The
estimation of $R_{i, t}$ uses incidence data and is done using the
EpiEstim package. Figure~\ref{fig:r_comp} shows the
reproduction number estimated using previous 28 days incidence
data. The high degree of correlation in the estimates from the three
different data sources shows that the estimation procedure is robust
to slight variations in reported number.


\begin{figure}
  \centering
  \includegraphics[width=\textwidth]{ms6-figures/who_vs_hm_vs_pm-R}
  \caption{Comparison of the reproduction numbers estimated from
        the different sources of the incidence data.}
  \label{fig:r_comp}
\end{figure}


In the interest of simplicity, we assume both
$\alpha$ and $\beta$ to be $1$. The other two parameters are
$p_{stay}$ and $\gamma$. We explored the influence of these two
parameters on the quality of fit of the predictions from the models at various points in the
epidemic. To assess the goodness-of-fit, we used the normalised root mean squared
error (rms), which is the sum of squares of the differences between
observed and predicted values. That is, 
\[ rms := \sum_{i = 1}^n{\left(o_i - p_i\right)^2},\]
where $o_i$ is $i$th observation, $p_i$ is the corresponding value
predicted by the model and $n$ is the total number of observations.

\begin{figure}
  \centering
  \begin{subfigure}[b]{0.4\textwidth}
    \includegraphics[]{ms6-figures/rms-100-2}
    \caption{Root mean square errors for prediction for 5 weeks at 100 days from the start of the epidemic.}
    \label{fig:rms-100}
\end{subfigure}
~
  \begin{subfigure}[b]{0.4\textwidth}
    \includegraphics[]{ms6-figures/rms-300-2}
    \caption{Root mean square errors for prediction for 5 weeks at 300 days from the start of the epidemic.}    
    \label{fig:rms-300}
\end{subfigure}
  \caption[RMS as a function of model parameters]{Normalised root mean
    squared error as a function of the model parameters. The fit is
    assessed for prediction of 5 weeks at 100 and 300 days from the start of
    the epidemic. The fit is better for smaller values of the root
    mean square error. In the early phase of the epidemic, a better
    fit is obtained at a smaller value of $p_{stay}$ while at the 300
    days mark, a much higher value of $p_{stay}$ is needed to obtain a
  good fit.}
  \label{fig:rms}
\end{figure}

\subsection{Predicting Future Cases}\label{predicting-future-cases}

The WHO data consisted of incidence data at the district level. To
validate the model, we carried out analysis at both the district level
and country level.

\subsubsection{Prediction at country level}

\subsubsection{Prediction at district level}
%\subsection{Connectivity and Risk of International Spread}\label{connectivity-and-risk-of-international-spread}

%\subsection{Parameters Inference}

\section{Conclusions and next steps}\label{sec:conclusions}







\newpage
\section*{References}
\bibliography{ms6}


\end{document}
