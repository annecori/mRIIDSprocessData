\documentclass[a4paper,12pt]{article}
\usepackage[utf8]{inputenc}
\usepackage{amsmath,amssymb,amsthm}
\usepackage{natbib}
\usepackage[margin=3cm]{geometry}
\usepackage{todonotes}
\usepackage{mathtools}
\usepackage{subfiles}
\usepackage{subcaption}
\usepackage{booktabs}

\begin{document}
\section{Information in the pop-up}
The expected number of cases in a place at a given time is estimated
using the observed number of cases in the past, pathogen characteristics (effective reproduction number
and serial interval) and the population movement between
countries. See About for details.

\section{About}

The expected number of cases in a place at a given time is estimated
using the observed number of cases in the past, pathogen characteristics (effective reproduction number
and serial interval) and the movement of populations between
countries.
\subsection*{Model}
The expected number of cases in a country at a given time is modelled as
a Poisson process. The mean of the process depends on past local incidence
as well as importations from other countries. The incidence count in a
given country is thus the sum of the cases linked to local
transmission and imported cases.

The number of infected individuals in a country $i$ at time $t$ is proportional
to the number of individuals infected at any point between the start
of the epidemic up to time $t$, multiplied by their
infectiousness. The distribution of the serial interval is used as the
infectiousness profile over time.
Thus, the number of infected individuals at time $t$
at country $i$, $I_{i, t}$, is approximated by the term
\[
  \sum_{s = 1}^{t}{\left( I_{i, t-s} \omega_s\right)}.
  \]

This number is then multiplied with the effective
reproduction number $R_{i, t}$, which is a measure of the
average number of secondary cases generated by a single primary
case.
The model also accounts for the movement into country $j$
from all countries $i$ by incorporating the probability of movement
from $i$ to $j$ ($p_{i \rightarrow j}$). This probability can be
estimated from different data sources e.g., flight data between
countries, and models of human population movement.

By summing over all countries under consideration,  we obtain an
estimated average rate of new cases in country $j$ in a given interval. Mathematically
this can be formulated as follows: the number of cases in country $j$ at time $t$ is
distributed according to a Poisson distribution with mean $\lambda_{j,
  t}$ given by
\[
  \lambda_{j, t} = \sum_{i = 1}^{n}{\left( p_{i \rightarrow j}R_{i,
        t}\sum_{s = 1}^{t}{\left( I_{i, t-s} \omega_s\right)}
    \right)}.
\]

\subsection*{Uncertainty in parameters and projections}

The model formulated above depends on parameters such as the serial
interval, reproduction number and the probability of movement between
countries. Since these
parameters cannot be perfectly observed, there is some uncertainty
associated with the values assigned to them. Furthermore, the
projected incidence counts are random samples from the Poisson distribution described above.

To account for the various sources of uncertainty, the model is run multiple times for a range of plausible values of each
parameter. The output of a single run of the model (simulation) is a
drawn from the  Poisson distribution. The multiple runs thus produce a ``cloud'' of
projected values. For each time point, the projections are reported as an interval
that contains 95\% of the projected values from several
simulations. The width of this interval is a measure of the uncertainty
associated with the forecasts.
\end{document}