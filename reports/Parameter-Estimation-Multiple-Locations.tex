\documentclass[11pt,]{article}
\usepackage[]{mathpazo}
\usepackage{amssymb,amsmath}
\usepackage{ifxetex,ifluatex}
\usepackage{fixltx2e} % provides \textsubscript
\ifnum 0\ifxetex 1\fi\ifluatex 1\fi=0 % if pdftex
  \usepackage[T1]{fontenc}
  \usepackage[utf8]{inputenc}
\else % if luatex or xelatex
  \ifxetex
    \usepackage{mathspec}
  \else
    \usepackage{fontspec}
  \fi
  \defaultfontfeatures{Ligatures=TeX,Scale=MatchLowercase}
\fi
% use upquote if available, for straight quotes in verbatim environments
\IfFileExists{upquote.sty}{\usepackage{upquote}}{}
% use microtype if available
\IfFileExists{microtype.sty}{%
\usepackage{microtype}
\UseMicrotypeSet[protrusion]{basicmath} % disable protrusion for tt fonts
}{}
\usepackage[margin=1in]{geometry}
\usepackage{hyperref}
\hypersetup{unicode=true,
            pdftitle={Estimation of Model Parameters},
            pdfborder={0 0 0},
            breaklinks=true}
\urlstyle{same}  % don't use monospace font for urls
\usepackage{natbib}
\bibliographystyle{apsr}
\usepackage{color}
\usepackage{fancyvrb}
\newcommand{\VerbBar}{|}
\newcommand{\VERB}{\Verb[commandchars=\\\{\}]}
\DefineVerbatimEnvironment{Highlighting}{Verbatim}{commandchars=\\\{\}}
% Add ',fontsize=\small' for more characters per line
\usepackage{framed}
\definecolor{shadecolor}{RGB}{248,248,248}
\newenvironment{Shaded}{\begin{snugshade}}{\end{snugshade}}
\newcommand{\KeywordTok}[1]{\textcolor[rgb]{0.13,0.29,0.53}{\textbf{#1}}}
\newcommand{\DataTypeTok}[1]{\textcolor[rgb]{0.13,0.29,0.53}{#1}}
\newcommand{\DecValTok}[1]{\textcolor[rgb]{0.00,0.00,0.81}{#1}}
\newcommand{\BaseNTok}[1]{\textcolor[rgb]{0.00,0.00,0.81}{#1}}
\newcommand{\FloatTok}[1]{\textcolor[rgb]{0.00,0.00,0.81}{#1}}
\newcommand{\ConstantTok}[1]{\textcolor[rgb]{0.00,0.00,0.00}{#1}}
\newcommand{\CharTok}[1]{\textcolor[rgb]{0.31,0.60,0.02}{#1}}
\newcommand{\SpecialCharTok}[1]{\textcolor[rgb]{0.00,0.00,0.00}{#1}}
\newcommand{\StringTok}[1]{\textcolor[rgb]{0.31,0.60,0.02}{#1}}
\newcommand{\VerbatimStringTok}[1]{\textcolor[rgb]{0.31,0.60,0.02}{#1}}
\newcommand{\SpecialStringTok}[1]{\textcolor[rgb]{0.31,0.60,0.02}{#1}}
\newcommand{\ImportTok}[1]{#1}
\newcommand{\CommentTok}[1]{\textcolor[rgb]{0.56,0.35,0.01}{\textit{#1}}}
\newcommand{\DocumentationTok}[1]{\textcolor[rgb]{0.56,0.35,0.01}{\textbf{\textit{#1}}}}
\newcommand{\AnnotationTok}[1]{\textcolor[rgb]{0.56,0.35,0.01}{\textbf{\textit{#1}}}}
\newcommand{\CommentVarTok}[1]{\textcolor[rgb]{0.56,0.35,0.01}{\textbf{\textit{#1}}}}
\newcommand{\OtherTok}[1]{\textcolor[rgb]{0.56,0.35,0.01}{#1}}
\newcommand{\FunctionTok}[1]{\textcolor[rgb]{0.00,0.00,0.00}{#1}}
\newcommand{\VariableTok}[1]{\textcolor[rgb]{0.00,0.00,0.00}{#1}}
\newcommand{\ControlFlowTok}[1]{\textcolor[rgb]{0.13,0.29,0.53}{\textbf{#1}}}
\newcommand{\OperatorTok}[1]{\textcolor[rgb]{0.81,0.36,0.00}{\textbf{#1}}}
\newcommand{\BuiltInTok}[1]{#1}
\newcommand{\ExtensionTok}[1]{#1}
\newcommand{\PreprocessorTok}[1]{\textcolor[rgb]{0.56,0.35,0.01}{\textit{#1}}}
\newcommand{\AttributeTok}[1]{\textcolor[rgb]{0.77,0.63,0.00}{#1}}
\newcommand{\RegionMarkerTok}[1]{#1}
\newcommand{\InformationTok}[1]{\textcolor[rgb]{0.56,0.35,0.01}{\textbf{\textit{#1}}}}
\newcommand{\WarningTok}[1]{\textcolor[rgb]{0.56,0.35,0.01}{\textbf{\textit{#1}}}}
\newcommand{\AlertTok}[1]{\textcolor[rgb]{0.94,0.16,0.16}{#1}}
\newcommand{\ErrorTok}[1]{\textcolor[rgb]{0.64,0.00,0.00}{\textbf{#1}}}
\newcommand{\NormalTok}[1]{#1}
\usepackage{graphicx,grffile}
\makeatletter
\def\maxwidth{\ifdim\Gin@nat@width>\linewidth\linewidth\else\Gin@nat@width\fi}
\def\maxheight{\ifdim\Gin@nat@height>\textheight\textheight\else\Gin@nat@height\fi}
\makeatother
% Scale images if necessary, so that they will not overflow the page
% margins by default, and it is still possible to overwrite the defaults
% using explicit options in \includegraphics[width, height, ...]{}
\setkeys{Gin}{width=\maxwidth,height=\maxheight,keepaspectratio}
\IfFileExists{parskip.sty}{%
\usepackage{parskip}
}{% else
\setlength{\parindent}{0pt}
\setlength{\parskip}{6pt plus 2pt minus 1pt}
}
\setlength{\emergencystretch}{3em}  % prevent overfull lines
\providecommand{\tightlist}{%
  \setlength{\itemsep}{0pt}\setlength{\parskip}{0pt}}
\setcounter{secnumdepth}{0}
% Redefines (sub)paragraphs to behave more like sections
\ifx\paragraph\undefined\else
\let\oldparagraph\paragraph
\renewcommand{\paragraph}[1]{\oldparagraph{#1}\mbox{}}
\fi
\ifx\subparagraph\undefined\else
\let\oldsubparagraph\subparagraph
\renewcommand{\subparagraph}[1]{\oldsubparagraph{#1}\mbox{}}
\fi

%%% Use protect on footnotes to avoid problems with footnotes in titles
\let\rmarkdownfootnote\footnote%
\def\footnote{\protect\rmarkdownfootnote}

%%% Change title format to be more compact
\usepackage{titling}

% Create subtitle command for use in maketitle
\newcommand{\subtitle}[1]{
  \posttitle{
    \begin{center}\large#1\end{center}
    }
}

\setlength{\droptitle}{-2em}
  \title{Estimation of Model Parameters}
  \pretitle{\vspace{\droptitle}\centering\huge}
  \posttitle{\par}
  \author{true}
  \preauthor{\centering\large\emph}
  \postauthor{\par}
  \predate{\centering\large\emph}
  \postdate{\par}
  \date{Y}


\begin{document}
\maketitle

\begin{Shaded}
\begin{Highlighting}[]
\KeywordTok{library}\NormalTok{(magrittr)}
\KeywordTok{library}\NormalTok{(ggthemes)}
\KeywordTok{library}\NormalTok{(ggplot2)}
\KeywordTok{library}\NormalTok{(dplyr)}
\KeywordTok{library}\NormalTok{(rstan)}
\KeywordTok{library}\NormalTok{(EpiEstim)}
\NormalTok{devtools}\OperatorTok{::}\KeywordTok{load_all}\NormalTok{()}
\end{Highlighting}
\end{Shaded}

\begin{verbatim}
## Loading mRIIDS
\end{verbatim}

\section{Parameter estimation for simulated
data}\label{parameter-estimation-for-simulated-data}

Load in the data and prepare to feed into Stan.

\begin{Shaded}
\begin{Highlighting}[]
\NormalTok{mean_SI  <-}\StringTok{ }\FloatTok{14.2}
\NormalTok{CV_SI    <-}\StringTok{ }\FloatTok{9.6} \OperatorTok{/}\StringTok{ }\FloatTok{14.2}
\NormalTok{SItrunc  <-}\StringTok{ }\DecValTok{40}
\NormalTok{SI_Distr <-}\StringTok{ }\KeywordTok{sapply}\NormalTok{(}\DecValTok{0}\OperatorTok{:}\NormalTok{SItrunc,}
                   \ControlFlowTok{function}\NormalTok{(e) }\KeywordTok{DiscrSI}\NormalTok{(e,}
\NormalTok{                                       mean_SI,}
\NormalTok{                                       mean_SI }\OperatorTok{*}\StringTok{ }\NormalTok{CV_SI))}
\NormalTok{SI_Distr <-}\StringTok{ }\NormalTok{SI_Distr }\OperatorTok{/}\StringTok{ }\KeywordTok{sum}\NormalTok{(SI_Distr)}
\end{Highlighting}
\end{Shaded}

\section{Multiple locations}\label{multiple-locations}

\subsection{Model}\label{model}

Incidence at location \(i\) at time \(t\) is given by \[
  I_{i, t} \sim Poisson(\lambda_{i, t}),
\] where \[
 \lambda_{i, t} = \sum_{i = 1}^{n}{p_{i, j}R_{i, t}
                                   \sum_{s = 1}^{t}{I_{i, s} 
                                   \omega_{t - s + 1}}}.
\]

\subsection{Simulated Data}\label{simulated-data}

\subsubsection{Example 1}\label{example-1}

\begin{Shaded}
\begin{Highlighting}[]
\NormalTok{n_loc  <-}\StringTok{ }\DecValTok{6}
\NormalTok{n_days <-}\StringTok{ }\DecValTok{120}
\NormalTok{I0     <-}\StringTok{ }\KeywordTok{matrix}\NormalTok{(}\KeywordTok{sample}\NormalTok{(}\DecValTok{10}\OperatorTok{:}\DecValTok{100}\NormalTok{, n_loc, }\DataTypeTok{replace =}\NormalTok{ T),}
                 \DataTypeTok{nrow =} \DecValTok{1}\NormalTok{)}



\NormalTok{change_at <-}\StringTok{ }\KeywordTok{seq}\NormalTok{(}\DataTypeTok{from =} \DecValTok{29}\NormalTok{, }\DataTypeTok{to =}\NormalTok{ n_days, }\DataTypeTok{by =} \DecValTok{28}\NormalTok{)}
\KeywordTok{set.seed}\NormalTok{(}\DecValTok{42}\NormalTok{)}
\NormalTok{Rjt1    <-}\StringTok{ }\KeywordTok{runif}\NormalTok{((n_loc }\OperatorTok{*}\StringTok{ }\KeywordTok{length}\NormalTok{( change_at))}\OperatorTok{/}\DecValTok{2}\NormalTok{,}
                 \DataTypeTok{min =} \DecValTok{0}\NormalTok{, }\DataTypeTok{max =} \DecValTok{2}\NormalTok{) }
\NormalTok{Rjt2    <-}\StringTok{ }\KeywordTok{runif}\NormalTok{((n_loc }\OperatorTok{*}\StringTok{ }\KeywordTok{length}\NormalTok{( change_at))}\OperatorTok{/}\DecValTok{2}\NormalTok{,}
                 \DataTypeTok{min =} \DecValTok{0}\NormalTok{, }\DataTypeTok{max =} \DecValTok{1}\NormalTok{) }
\NormalTok{Rjt    <-}\StringTok{ }\KeywordTok{c}\NormalTok{(Rjt1, Rjt2)}
\NormalTok{R_sim  <-}\StringTok{ }\KeywordTok{makeRmatrix}\NormalTok{( Rjt,}
                       \DataTypeTok{ncol =}\NormalTok{ n_loc,}
                       \DataTypeTok{nrow =}\NormalTok{ n_days }\OperatorTok{+}\StringTok{ }\KeywordTok{nrow}\NormalTok{(I0),}
                       \DataTypeTok{change_at =}\NormalTok{ change_at)}
\end{Highlighting}
\end{Shaded}

\begin{verbatim}
## Warning in mapply(rep, x = split_R, times = num_rows): longer argument not
## a multiple of length of shorter
\end{verbatim}

\begin{Shaded}
\begin{Highlighting}[]
\CommentTok{#pij <- matrix(c(0.9, 0.06, 0.08,}
\CommentTok{#                0.03, 0.9, 0.02,}
\CommentTok{#                0.07, 0.04, 0.9), nrow = 3, ncol = 3)}

\NormalTok{pij <-}\StringTok{ }\KeywordTok{runif}\NormalTok{(n_loc }\OperatorTok{*}\StringTok{ }\NormalTok{n_loc)}
\NormalTok{pij <-}\StringTok{ }\KeywordTok{matrix}\NormalTok{(pij, }\DataTypeTok{nrow =}\NormalTok{ n_loc)}
\NormalTok{pij <-}\StringTok{ }\NormalTok{pij}\OperatorTok{/}\KeywordTok{rowSums}\NormalTok{(pij)}

\NormalTok{I <-}\StringTok{ }\KeywordTok{project2}\NormalTok{(}\DataTypeTok{incid =}\NormalTok{ I0, }\DataTypeTok{R =}\NormalTok{ R_sim, }\DataTypeTok{si =}\NormalTok{ SI_Distr,}
              \DataTypeTok{pij =}\NormalTok{ pij,}
              \DataTypeTok{n.days =}\NormalTok{ n_days)}
\end{Highlighting}
\end{Shaded}

Preparing data for feeding into Stan.

\begin{Shaded}
\begin{Highlighting}[]
\NormalTok{I  <-}\StringTok{ }\KeywordTok{rbind}\NormalTok{( I0, I)}
\NormalTok{T  <-}\StringTok{ }\KeywordTok{nrow}\NormalTok{(I)}
\NormalTok{N  <-}\StringTok{ }\KeywordTok{ncol}\NormalTok{(I)}

\NormalTok{SI <-}\StringTok{ }\NormalTok{SI_Distr[ }\DecValTok{1}\OperatorTok{:}\NormalTok{( T }\OperatorTok{+}\StringTok{ }\DecValTok{1}\NormalTok{)]}
\NormalTok{SI[ }\KeywordTok{which}\NormalTok{( }\KeywordTok{is.na}\NormalTok{(SI))] <-}\StringTok{ }\DecValTok{0}

\NormalTok{change_at <-}\StringTok{ }\NormalTok{change_at }\OperatorTok{+}\StringTok{ }\DecValTok{35}
\NormalTok{change_at <-}\StringTok{ }\NormalTok{change_at[}\KeywordTok{which}\NormalTok{(change_at }\OperatorTok{<}\StringTok{ }\NormalTok{T)]}
\NormalTok{num_Rjt   <-}\StringTok{  }\NormalTok{n_loc }\OperatorTok{*}\StringTok{ }\NormalTok{(}\KeywordTok{length}\NormalTok{( change_at) }\OperatorTok{+}\StringTok{ }\DecValTok{1}\NormalTok{)}
\NormalTok{Rjt_stan  <-}\StringTok{ }\KeywordTok{seq}\NormalTok{(num_Rjt)}
\NormalTok{rindex    <-}\StringTok{ }\KeywordTok{makeRmatrix}\NormalTok{(}\KeywordTok{seq}\NormalTok{(Rjt_stan), }\DataTypeTok{nrow =}\NormalTok{ T, }\DataTypeTok{ncol =}\NormalTok{ N,}
                      \DataTypeTok{change_at =}\NormalTok{ change_at)}

\NormalTok{sim_data <-}\StringTok{ }\KeywordTok{list}\NormalTok{(}\DataTypeTok{T =}\NormalTok{ T, }\DataTypeTok{N =}\NormalTok{ N, }\DataTypeTok{I =}\NormalTok{ I, }\DataTypeTok{SI =}\NormalTok{ SI,}
                 \DataTypeTok{rindex =}\NormalTok{ rindex,}
                 \DataTypeTok{num_Rjt =}\NormalTok{ num_Rjt, }\DataTypeTok{pstay =}\NormalTok{ pij)}

\NormalTok{model_file <-}\StringTok{ }\NormalTok{here}\OperatorTok{::}\KeywordTok{here}\NormalTok{(}\StringTok{"R"}\NormalTok{, }\StringTok{"multiple_location.stan"}\NormalTok{)}
\NormalTok{fit1 <-}\StringTok{ }\KeywordTok{stan}\NormalTok{(}
  \DataTypeTok{file =}\NormalTok{ model_file,  }
  \DataTypeTok{data =}\NormalTok{ sim_data,   }
  \DataTypeTok{chains =} \DecValTok{2}\NormalTok{,        }
  \DataTypeTok{warmup =} \DecValTok{1000}\NormalTok{,     }
  \DataTypeTok{iter =} \DecValTok{2000}\NormalTok{,       }
  \DataTypeTok{cores =} \DecValTok{2}\NormalTok{,         }
  \DataTypeTok{refresh =} \DecValTok{500}\NormalTok{)     }

\KeywordTok{print}\NormalTok{(fit1, }\DataTypeTok{pars=}\KeywordTok{c}\NormalTok{(}\StringTok{"R"}\NormalTok{), }\DataTypeTok{probs=}\KeywordTok{c}\NormalTok{(.}\DecValTok{01}\NormalTok{, .}\DecValTok{1}\NormalTok{,.}\DecValTok{5}\NormalTok{,.}\DecValTok{9}\NormalTok{))}
\end{Highlighting}
\end{Shaded}

\begin{verbatim}
## Inference for Stan model: multiple_location.
## 2 chains, each with iter=2000; warmup=1000; thin=1; 
## post-warmup draws per chain=1000, total post-warmup draws=2000.
## 
##       mean se_mean   sd   1%  10%  50%  90% n_eff Rhat
## R[1]  1.25    0.00 0.14 0.92 1.08 1.25 1.43  1233    1
## R[2]  0.15    0.00 0.13 0.00 0.02 0.12 0.33  2000    1
## R[3]  0.23    0.00 0.16 0.01 0.04 0.20 0.45  2000    1
## R[4]  0.97    0.01 0.25 0.41 0.65 0.97 1.30  1338    1
## R[5]  1.29    0.01 0.19 0.83 1.03 1.28 1.53  1476    1
## R[6]  2.74    0.01 0.34 1.99 2.30 2.73 3.17  1446    1
## R[7]  0.83    0.00 0.18 0.41 0.60 0.84 1.06  2000    1
## R[8]  0.36    0.01 0.26 0.01 0.07 0.32 0.72  2000    1
## R[9]  0.76    0.01 0.33 0.08 0.34 0.75 1.18  2000    1
## R[10] 0.27    0.00 0.20 0.00 0.05 0.23 0.56  2000    1
## R[11] 0.98    0.01 0.22 0.46 0.69 0.97 1.28  2000    1
## R[12] 1.18    0.02 0.86 0.02 0.20 1.01 2.38  1359    1
## R[13] 0.30    0.00 0.21 0.00 0.05 0.26 0.59  2000    1
## R[14] 1.87    0.02 0.80 0.23 0.82 1.87 2.90  2000    1
## R[15] 0.62    0.01 0.40 0.01 0.13 0.58 1.18  2000    1
## R[16] 0.53    0.01 0.37 0.01 0.09 0.47 1.04  2000    1
## R[17] 0.49    0.01 0.31 0.01 0.11 0.44 0.92  2000    1
## R[18] 1.47    0.03 0.94 0.02 0.31 1.36 2.78  1242    1
## R[19] 1.59    0.03 1.13 0.04 0.31 1.40 3.13  2000    1
## R[20] 1.77    0.03 1.54 0.02 0.23 1.32 4.01  2000    1
## R[21] 2.24    0.04 1.85 0.02 0.24 1.83 4.89  2000    1
## R[22] 1.29    0.02 1.05 0.01 0.18 1.07 2.76  2000    1
## R[23] 1.92    0.03 1.42 0.03 0.34 1.64 3.88  2000    1
## R[24] 2.43    0.05 2.04 0.03 0.33 1.93 5.29  2000    1
## 
## Samples were drawn using NUTS(diag_e) at Tue Feb 13 13:46:39 2018.
## For each parameter, n_eff is a crude measure of effective sample size,
## and Rhat is the potential scale reduction factor on split chains (at 
## convergence, Rhat=1).
\end{verbatim}

\subsection{Test model fit}\label{test-model-fit}

The fitted model has draws from the posterior distribution. We can use
these samples to project forward.

\begin{Shaded}
\begin{Highlighting}[]
\NormalTok{list_of_draws <-}\StringTok{ }\NormalTok{rstan}\OperatorTok{::}\KeywordTok{extract}\NormalTok{(fit1)}
\NormalTok{r_samples <-}\StringTok{ }\NormalTok{list_of_draws[[}\StringTok{"R"}\NormalTok{]]}
\NormalTok{dates <-}\StringTok{ }\KeywordTok{seq}\NormalTok{(}\DataTypeTok{from =} \KeywordTok{as.Date}\NormalTok{(}\StringTok{"2018-02-07"}\NormalTok{), }\DataTypeTok{by =} \StringTok{"1 day"}\NormalTok{,}
             \DataTypeTok{length.out =} \KeywordTok{nrow}\NormalTok{(I))}

\NormalTok{daily_projections <-}\StringTok{ }\NormalTok{plyr}\OperatorTok{::}\KeywordTok{alply}\NormalTok{(r_samples,}
                                 \DecValTok{1}\NormalTok{, }
                                 \ControlFlowTok{function}\NormalTok{(r.t)\{}
\NormalTok{                                     R_posterior <-}
\StringTok{                                         }\KeywordTok{makeRmatrix}\NormalTok{(r.t,}
                                                     \DataTypeTok{nrow =} \KeywordTok{nrow}\NormalTok{(I),}
                                                     \DataTypeTok{ncol =}\NormalTok{ n_loc,}
                                                     \DataTypeTok{change_at =}\NormalTok{ change_at)                                     }
\NormalTok{                                     out <-}\StringTok{ }\KeywordTok{project2}\NormalTok{(I0,}
\NormalTok{                                                     R_posterior,}
\NormalTok{                                                     SI_Distr,}
\NormalTok{                                                     pij,}
\NormalTok{                                                     n_days) }
\NormalTok{                                     projected <-}\StringTok{ }\KeywordTok{rbind}\NormalTok{(I0,}
\NormalTok{                                                        out)}
\NormalTok{                                     projected }\OperatorTok
\StringTok{                                      }\KeywordTok{as.data.frame}\NormalTok{(.) }
                                     
\NormalTok{                                     projected}\OperatorTok{$}\NormalTok{Date <-}\StringTok{ }\NormalTok{dates}
                                     \KeywordTok{return}\NormalTok{(projected)\})}



\NormalTok{tmp <-}\StringTok{ }\KeywordTok{bind_rows}\NormalTok{(daily_projections)}
\NormalTok{projections_distr <-}\StringTok{ }\KeywordTok{projection_quantiles}\NormalTok{(tmp)}
\end{Highlighting}
\end{Shaded}

\begin{verbatim}
## Joining, by = c("Date", "Country")
## Joining, by = c("Date", "Country")
\end{verbatim}

\begin{Shaded}
\begin{Highlighting}[]
\NormalTok{I_df <-}\StringTok{ }\KeywordTok{as.data.frame}\NormalTok{(I) }
\NormalTok{I_df}\OperatorTok{$}\NormalTok{Date <-}\StringTok{ }\NormalTok{dates}
\NormalTok{I_df }\OperatorTok\StringTok{ }\NormalTok{tidyr}\OperatorTok{::}\KeywordTok{gather}\NormalTok{(}\StringTok{"Country"}\NormalTok{, }\StringTok{"Incidence"}\NormalTok{, }\OperatorTok{-}\NormalTok{Date)}


\NormalTok{I_df}\OperatorTok{$}\NormalTok{Date }\OperatorTok\StringTok{ }\NormalTok{as.Date}
\NormalTok{p   <-}\StringTok{ }\KeywordTok{ggplot}\NormalTok{(I_df, }\KeywordTok{aes}\NormalTok{(Date, Incidence)) }\OperatorTok{+}
\StringTok{           }\KeywordTok{geom_point}\NormalTok{(}\DataTypeTok{size =} \DecValTok{2}\NormalTok{, }\DataTypeTok{stroke =} \DecValTok{0}\NormalTok{, }\DataTypeTok{shape =} \DecValTok{16}\NormalTok{) }\OperatorTok{+}
\StringTok{           }\KeywordTok{facet_wrap}\NormalTok{(}\OperatorTok{~}\NormalTok{Country, }\DataTypeTok{scales =} \StringTok{"free_y"}\NormalTok{)}
\NormalTok{p <-}\StringTok{ }\NormalTok{p }\OperatorTok{+}\StringTok{ }\KeywordTok{geom_line}\NormalTok{(}\DataTypeTok{data =}\NormalTok{ projections_distr,}
                   \KeywordTok{aes}\NormalTok{(}\DataTypeTok{x =}\NormalTok{ Date, }\DataTypeTok{y =}\NormalTok{ y, }\DataTypeTok{group =} \DecValTok{1}\NormalTok{),}
                     \DataTypeTok{color =} \StringTok{'black'}\NormalTok{, }\DataTypeTok{size =} \FloatTok{0.9}\NormalTok{)}
\NormalTok{p <-}\StringTok{ }\NormalTok{p }\OperatorTok{+}\StringTok{ }\KeywordTok{geom_ribbon}\NormalTok{(}\DataTypeTok{data =}\NormalTok{ projections_distr, }\KeywordTok{aes}\NormalTok{(}\DataTypeTok{x =}\NormalTok{ Date,}
                                                   \DataTypeTok{ymin =}\NormalTok{ ymin,}
                                                   \DataTypeTok{ymax =}\NormalTok{ ymax,}
                                                   \DataTypeTok{group =} \DecValTok{1}\NormalTok{),}
                     \DataTypeTok{inherit.aes =} \OtherTok{FALSE}\NormalTok{,}
                     \DataTypeTok{alpha =} \FloatTok{0.5}\NormalTok{)}

\NormalTok{outname <-}\StringTok{ }\KeywordTok{paste0}\NormalTok{(}\KeywordTok{Sys.Date}\NormalTok{(), }\StringTok{"-"}\NormalTok{, N, }\StringTok{"-"}\NormalTok{, T, }\StringTok{".pdf"}\NormalTok{)}
\NormalTok{outfile <-}\StringTok{ }\NormalTok{here}\OperatorTok{::}\KeywordTok{here}\NormalTok{(}\StringTok{"output/figures"}\NormalTok{, outname)}
\KeywordTok{ggsave}\NormalTok{(outfile, p)}
\end{Highlighting}
\end{Shaded}

\begin{verbatim}
## Saving 6.5 x 4.5 in image
\end{verbatim}


\end{document}
