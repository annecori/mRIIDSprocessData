\documentclass[11pt,]{article}
\usepackage[]{mathpazo}
\usepackage{amssymb,amsmath}
\usepackage{subcaption}
\usepackage[margin=1in]{geometry}
\usepackage{graphicx}
\usepackage[below]{placeins}
\usepackage[utf8]{inputenc}
\usepackage[T1]{fontenc}
\usepackage{natbib}
\bibliographystyle{abbrv}
\usepackage{url}
\title{Estimating the relative risk of spatial spread of the Ebola
  outbreak in and around DRC}
\author{Sangeeta Bhatia, Anne Cori, Christl A Donnelly,
   Natsuko Imai, Pierre Nouvellet}
\begin{document}
\maketitle
This document presents an overview of analyses conducted to estimate the relative
risk of importation of the current Ebola outbreak into the unaffected
regions of
Democratic Republic of the Congo (DRC) and neighbouring countries.

\section*{Methods}

\subsection*{General approach}

In previous work, we have developed reliable and tested methods
\citep{cori2013new, who2014ebola, nouvellet2017simple} to produce
short term incidence forecasts.
Those methods relie on a minimal amount of data: a time series of
past incident cases in a particular area, and the typical infectivity profile once
symptoms are developed (approximated by the serial interval).
The methods essentially infer the transmissibility (i.e. the
reproduction number or average number of secondary case linked to a
primary case)
and then project incidence forward assuming the transmissibility
remains constant.
While this forecasting method is powerful, it does not account for cases moving from one geographical location to another.

Here, we build on those methods to include spatial spread of the
disease. i.e. forecasted cases can move from one area to another.

An important methodological challenge is to quantify the flow of
populations between geographical areas.
We used the general framework of gravity models to estimate such flow.

Under a gravity model, the flow of individuals from area \(i\) to area \(j\),
\(\phi_{i \rightarrow j}\), is proportional to the product of the
populations of the two areas, \(N_i\) and \(N_j\) and inversely
proportional to the distance between them \(d_{i, j}\), all quantities
are raised to some power.
\[
  \phi_{i \rightarrow j} = K \frac{N_i^{\alpha}N_j^{\beta}}{d_{i, j}^{\gamma}}.
\]

In the current analysis, \(K\), \( \alpha \), \( \beta \) and \(\gamma\) 
are assumed to be $1$. However, if we had sufficient data regarding
the flows of populations, we could estimate those parameters.
We are also currently reviewing the literature to find relevant
estimates of those parameters for DRC and neighbouring countries,
as well as exploring the use of other models of human population movements such as radiation model. 

We then assume that the probability of importation of a case from a location \(i\) to another location \(j\) is
proportional to the relative flow of population from \(i\) to \(j\).
Therefore, the relative risk of spread \(r_{i \rightarrow j}\)
at a location \(j\) from a location \(i\)
is proportional to the relative population flow into location \(j\) from location \(i\).

\[
  r_{i \rightarrow j} = \frac{\phi_{i \rightarrow
  j}}{\sum_{x}{\phi_{i \rightarrow
  x}}}.
\]

\FloatBarrier



The overall risk at a location \( j \) is the 
sum over all possible sources \( i \) weighted by the current
infectivity at \(i\). Thus
\[ r_j = \sum_{i}{\lambda_i r_{i \rightarrow j}}.\]

where \(\lambda_i\) is the infectivity at \(i\).
Overall infectivity \( \lambda_i \) at a location \(i\) at a given time
\(t\) is the infectivity due to previously infected individuals.
This takes into account the number of individuals infected in the past and their infectiousness profile over time. That is,

\[ \lambda_i(t) = \sum_{k = 1}^{t}{I_{t - k} \omega_k},\]
where \( \omega_k \) is the relative infectivity at time \( k \) (as measured by serial interval distribution).


In the current outbreak, all reported cases were located in two ADM2 regions 
Équateur and Mbandaka (Figure~\ref{fig:infprofile}).
\footnote{The boundaries of the administrative divisions in DRC varied
  across the different data sources that we could access. The
  boundaries shown in the maps in this report may not agree with the
  current boundaries in DRC and we apologise for the discrepancies.
  In particular, in this report Équateur and Mbandaka are both
  referred to as ADM2 regions. These regions have been highlighted in the map
  in Figure~\ref{fig:infprofile} for clarification.  
}
The estimated risk \(r_j\) at location \(j\) is thus the sum of
the risk of importation from Équateur and Mbandaka, weighted by the
overall infectivity at the two locations (Figure~\ref{fig:infprofile}).  
\[
  r_j = \lambda_{equateur} r_{equateur \rightarrow j} +
  \lambda_{mbandaka} r_{mbandaka \rightarrow j}. 
\]


The incidence time series used to predict the risk and
the infectiousness profile are shown in Figure~\ref{fig:infprofile}.
We used the infectivity on 21\textsuperscript{st} May as weights to
estimate the overall risk.

\begin{figure}
  \centering
  \includegraphics[width = 0.8\textwidth]{who_report_fig1}
  \caption{The top left figure shows the incidence time series in the
    current outbreak in DRC up to
    21 May 2018 stratified by location. The serial interval distribution is shown on the
    bottom left. These data are used to determine the relative overall
    infectivity in Équateur and Mbandaka (top right) that was used to
    estimate the weighted relative risk  of the spread of the 
    the outbreak within and outside DRC. Équateur and Mbandaka are
    highlighted in the map on the bottom right.}
  \label{fig:infprofile}  
\end{figure}

Following \citep{cori2013new}, we used the serial interval
distribution, defined as the time period between onset of symptoms in
an index case
and secondary case, as a proxy for the infectiousness profile over time.
The serial interval distribution was assumed to be Gamma distributed
with mean 15.3 days and standard deviation 9.1 days.
These values were taken from estimates of the West African epidemic \citep{who2014ebola}.

\subsection*{Analyses performed}

We considered the risk of importation into the unaffected regions of
DRC and the following countries: Angola, Burundi,
Central African Republic, Democratic Republic of the Congo (DRC), 
Republic of Congo, Rwanda, South Sudan,
Uganda, and Zambia. 

Within DRC, the risk of importation was characterised at ADM2
resolution because it was at this scale that data consistent across
the data sources we were using were available.

For neighbouring countries, we chose the resolution
(ADM0, ADM1 or ADM2) that most closely matched the ADM2 resolution in
DRC in term of either average population size or average surface area.

While assessing the risk of importation at ADM2 level in DRC is
critical, we felt that the risk to neighbouring countries
could also be assessed at the country level (in addition to smaller administrative units).
Therefore, for each neighbouring country, we have also presented the risk of
importation aggregated
over the entire country to obtain an estimate of the risk of Ebola importation at the national scale.


\section*{Results}

Figure~\ref{fig:relrisk} presents the estimated relative risks.

\begin{figure}
  \centering
  \includegraphics[width = \textwidth]{relative_risks.pdf}
    \caption[]{The top left figure shows the risk at the spatial level
      used in the analyses and the top right figure shows the risk
      aggregated at the national level for countries neighbouring DRC.
      The bottom row shows the results of similar analysis
  where the resolution in each country was chosen to match average
  ADM2 spatial area in DRC rather than to match average population.
  In this case, we worked at ADM0 level in
  Burundi and Rwanda while for all other countries analysis was done
  at ADM1 level in both cases.
  Équateur  and Mbandaka are shown in dark gray.}
  \label{fig:relrisk}
\end{figure}
\FloatBarrier


Under the current parameterisation of this model, the effect of large population size
drives the risk estimates. For instance, the risk of importation in Luanda
(Angola) is high despite a large region of low risk lying between
Luanda and Équateur. This is because Luanda is the third most
heavily populated region in the countries considered. A similar trend
is observed when the risk is presented as aggregated over a country where Rwanda
and Angola are seen as high risk countries.

Therefore, our results are highly influenced by our choice of the
mobility model (gravity model) and the parameters used for the
model. A more reliable quantification of population movements
would lead to more reliable estimates of the risk of importations.

\section*{Caveats}

Our analysis relies on accurate meta-data (population
sizes, accurate co-ordinates and areas) about  
the administrative units (ADM1, ADM2 and ADM3).
These data are in turn used to estimate flow of populations.

Population sizes  of administrative units were extracted
from Landscan 2015 population density data combined with GADM 2.8 shape files. 

While the boundaries of the administrative units for all
countries (except DRC) were extracted from the Database of Global Administrative Areas (\url{https://gadm.org/}),
 the shapefiles of DRC were shared with us by our collaborator.
We encountered various issues where the names and format of data
between datasets were inconsistent.
This problem was particularly acute in DRC. We also had to exclude
Tanzania from our analysis because of issues such as multiple regions
having common names and different data sources having different codes
for regions.

Availability of standardised and approved shapefiles for different
administrative boundaries and latest census data would greatly facilitate our analysis.



\section*{Data Sources}
\begin{itemize}
\item Incidence trends were derived from the linelist \citep{who2018drc}. Only confirmed
  cases with onset of symptoms between 30\textsuperscript{th} April
  and 21\textsuperscript{st} May were used for this analysis. 
 \item Serial interval parameters were extracted from \citep{who2014ebola}.
 \item Shapefiles for all countries except DRC were downloaded from
   GADM. Shapefiles for DRC were shared by a collaborator.
 \item Population estimates were obtained from Landscan 2015
   population density data.
 \end{itemize}


\section*{Going forward}

Ideally, we would like to estimate pan-Africa  risk of importation.

To achieve this, we would require:
\begin{itemize}
 \item Shapefiles for all countries in Africa at ADM0, ADM1, ADM2 and ADM3,
 \item Corresponding estimates of population sizes at ADM0, ADM1, ADM2 and ADM3.
\end{itemize}

We would also need to improve our estimate of populations flow.
For this as we had mentioned before we would need information on:

\begin{itemize}
 \item Flow of people in and out of the administrative units defined
   above.
   Ideally, for each administrative units, this would take the form of:
 \item How many people enter these units (per unit time) and where they came from,
 \item How many people exit these units (per unit time) and where they are going.
 \end{itemize}
 
We realise this information will be available only for a subset of
geographical units and will quite likely not be, strictly speaking,
confined to geographical units.
However the more information we have on this (e.g. movement between 2
cities, ports),
the better we can predict population flow in locations where
information is missing.

The analysis presented in this report predicts the \emph{relative} risk of
importation. The above mentioned data would also help us estimate the
constant of proportionality \(K\) in the gravity model and absolute
flow of population between regions. This
information together with estimates of reproduction number (that we can
produce) would allow us to predict absolute risk of importation. In
future iterations of this work, the model can be enhanced to include
these changes.

Our general framework can also account for susceptibility to
disease across space and the local capacity and ability to contain an outbreak.
Although perhaps less critical, if data where available concerning the
above, we could envisage including such effects in our model.


 \bibliography{who}
\end{document}
