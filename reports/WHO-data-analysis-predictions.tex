\documentclass[11pt,]{article}
\usepackage[]{mathpazo}
\usepackage{amssymb,amsmath}
\usepackage{ifxetex,ifluatex}
\usepackage{fixltx2e} % provides \textsubscript
\ifnum 0\ifxetex 1\fi\ifluatex 1\fi=0 % if pdftex
  \usepackage[T1]{fontenc}
  \usepackage[utf8]{inputenc}
\else % if luatex or xelatex
  \ifxetex
    \usepackage{mathspec}
  \else
    \usepackage{fontspec}
  \fi
  \defaultfontfeatures{Ligatures=TeX,Scale=MatchLowercase}
\fi
% use upquote if available, for straight quotes in verbatim environments
\IfFileExists{upquote.sty}{\usepackage{upquote}}{}
% use microtype if available
\IfFileExists{microtype.sty}{%
\usepackage{microtype}
\UseMicrotypeSet[protrusion]{basicmath} % disable protrusion for tt fonts
}{}
\usepackage[margin=1in]{geometry}
\usepackage{hyperref}
\hypersetup{unicode=true,
            pdftitle={Forward projections using WHO incidence data},
            pdfborder={0 0 0},
            breaklinks=true}
\urlstyle{same}  % don't use monospace font for urls
\usepackage{natbib}
\bibliographystyle{apsr}
\usepackage{graphicx,grffile}
\makeatletter
\def\maxwidth{\ifdim\Gin@nat@width>\linewidth\linewidth\else\Gin@nat@width\fi}
\def\maxheight{\ifdim\Gin@nat@height>\textheight\textheight\else\Gin@nat@height\fi}
\makeatother
% Scale images if necessary, so that they will not overflow the page
% margins by default, and it is still possible to overwrite the defaults
% using explicit options in \includegraphics[width, height, ...]{}
\setkeys{Gin}{width=\maxwidth,height=\maxheight,keepaspectratio}
\IfFileExists{parskip.sty}{%
\usepackage{parskip}
}{% else
\setlength{\parindent}{0pt}
\setlength{\parskip}{6pt plus 2pt minus 1pt}
}
\setlength{\emergencystretch}{3em}  % prevent overfull lines
\providecommand{\tightlist}{%
  \setlength{\itemsep}{0pt}\setlength{\parskip}{0pt}}
\setcounter{secnumdepth}{0}
% Redefines (sub)paragraphs to behave more like sections
\ifx\paragraph\undefined\else
\let\oldparagraph\paragraph
\renewcommand{\paragraph}[1]{\oldparagraph{#1}\mbox{}}
\fi
\ifx\subparagraph\undefined\else
\let\oldsubparagraph\subparagraph
\renewcommand{\subparagraph}[1]{\oldsubparagraph{#1}\mbox{}}
\fi

%%% Use protect on footnotes to avoid problems with footnotes in titles
\let\rmarkdownfootnote\footnote%
\def\footnote{\protect\rmarkdownfootnote}

%%% Change title format to be more compact
\usepackage{titling}

% Create subtitle command for use in maketitle
\newcommand{\subtitle}[1]{
  \posttitle{
    \begin{center}\large#1\end{center}
    }
}

\setlength{\droptitle}{-2em}
  \title{Forward projections using WHO incidence data}
  \pretitle{\vspace{\droptitle}\centering\huge}
  \posttitle{\par}
  \author{true}
  \preauthor{\centering\large\emph}
  \postauthor{\par}
  \predate{\centering\large\emph}
  \postdate{\par}
  \date{10 April, 2018}


\begin{document}
\maketitle

Split the data into training and validation sets.

\section{Parameters for Ebola}\label{parameters-for-ebola}

Culled from literature.

\section{Gravity model parameters}\label{gravity-model-parameters}

\section{Estimating the reproduction
number}\label{estimating-the-reproduction-number}

For the predictions, we need to estimate R only in the training window.

Write the reproduction numbers out for future analysis.

We use the estimated R at t.proj to be the R for the window over which
predictions are being carried out.

Calculate the probability of movement between locations. There is some
extra arranging of variables here to ensure that the columns in the
incidence data and movement matrix are in the same order.

\subsection{Aggregating district
predictions}\label{aggregating-district-predictions}

We have \(N\) simulations for predictions. This means that for each
district we have \(N\) simulated numbers. To obtain the aggregated
national counts, for each date, draw a random sample of size \(k\) from
the \(N\) simulated numbers for each district. Shuffle each sample and
add.

And write them out.

\subsection{WHO weekly country data}\label{who-weekly-country-data}

\subsection{Quantiles for district
predictions}\label{quantiles-for-district-predictions}

\section{Forecast visualisation}\label{forecast-visualisation}

The plot consists of the following elements: the training and validation
sets, distribution of the predictions (quantiles) and some data beyond
the last date for which prediction has been carried out so that we can
see the trend.

\section{Evaluating Goodness-of-Fit}\label{evaluating-goodness-of-fit}

\subsection{Proportion of weekly incidence points that fall inside the
95\%
CI}\label{proportion-of-weekly-incidence-points-that-fall-inside-the-95-ci}

\subsection{Coefficient of
Determination}\label{coefficient-of-determination}

For \(k\) spatial units,

\[ R^2 = 1 - \frac{SS_{res}}{SS_{tot}},\]

where

\[ SS_{res} = \sum_{i = 1}^n{y_i - \hat{y_i}},\]

and \[ SS_{tot} = \sum_{k}{\sum_{i = 1}^n{y_{i, k} - \bar{y_k}}}.\]


\end{document}
