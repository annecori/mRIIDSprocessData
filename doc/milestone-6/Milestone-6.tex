\documentclass[11pt,]{article}
\usepackage[]{mathpazo}
\usepackage{amssymb,amsmath}
\usepackage{ifxetex,ifluatex}
\usepackage{fixltx2e} % provides \textsubscript
\ifnum 0\ifxetex 1\fi\ifluatex 1\fi=0 % if pdftex
  \usepackage[T1]{fontenc}
  \usepackage[utf8]{inputenc}
\else % if luatex or xelatex
  \ifxetex
    \usepackage{mathspec}
  \else
    \usepackage{fontspec}
  \fi
  \defaultfontfeatures{Ligatures=TeX,Scale=MatchLowercase}
\fi
% use upquote if available, for straight quotes in verbatim environments
\IfFileExists{upquote.sty}{\usepackage{upquote}}{}
% use microtype if available
\IfFileExists{microtype.sty}{%
\usepackage{microtype}
\UseMicrotypeSet[protrusion]{basicmath} % disable protrusion for tt fonts
}{}
\usepackage[margin=1in]{geometry}
\usepackage{hyperref}
\hypersetup{unicode=true,
            pdftitle={Milestone 6},
            pdfborder={0 0 0},
            breaklinks=true}
\urlstyle{same}  % don't use monospace font for urls
\usepackage{natbib}
\bibliographystyle{plainnat}
\usepackage{graphicx,grffile}
\makeatletter
\def\maxwidth{\ifdim\Gin@nat@width>\linewidth\linewidth\else\Gin@nat@width\fi}
\def\maxheight{\ifdim\Gin@nat@height>\textheight\textheight\else\Gin@nat@height\fi}
\makeatother
% Scale images if necessary, so that they will not overflow the page
% margins by default, and it is still possible to overwrite the defaults
% using explicit options in \includegraphics[width, height, ...]{}
\setkeys{Gin}{width=\maxwidth,height=\maxheight,keepaspectratio}
\IfFileExists{parskip.sty}{%
\usepackage{parskip}
}{% else
\setlength{\parindent}{0pt}
\setlength{\parskip}{6pt plus 2pt minus 1pt}
}
\setlength{\emergencystretch}{3em}  % prevent overfull lines
\providecommand{\tightlist}{%
  \setlength{\itemsep}{0pt}\setlength{\parskip}{0pt}}
\setcounter{secnumdepth}{5}
% Redefines (sub)paragraphs to behave more like sections
\ifx\paragraph\undefined\else
\let\oldparagraph\paragraph
\renewcommand{\paragraph}[1]{\oldparagraph{#1}\mbox{}}
\fi
\ifx\subparagraph\undefined\else
\let\oldsubparagraph\subparagraph
\renewcommand{\subparagraph}[1]{\oldsubparagraph{#1}\mbox{}}
\fi

%%% Use protect on footnotes to avoid problems with footnotes in titles
\let\rmarkdownfootnote\footnote%
\def\footnote{\protect\rmarkdownfootnote}

%%% Change title format to be more compact
\usepackage{titling}

% Create subtitle command for use in maketitle
\newcommand{\subtitle}[1]{
  \posttitle{
    \begin{center}\large#1\end{center}
    }
}

\setlength{\droptitle}{-2em}
  \title{Mapping the Risk of International Infectious Disease Spread
    (MRIIDS)}
  \subtitle{A project funded through USAID’s ``Combating Zika and
    Future Threats: A Grand Challenge for Development'' program \\
    Milestone 6: Increased complexity of the simple model to include data from Milestone 5}
  \pretitle{\vspace{\droptitle}\centering\huge}
  \posttitle{\par}
  \author{}
  \preauthor{}\postauthor{}
  \predate{\centering\large\emph}
  \postdate{\par}
  \date{27 November, 2017}

\usepackage{tikz}
\usetikzlibrary{positioning}
\usetikzlibrary{chains}

\begin{document}
\maketitle
\tableofcontents

\newpage
\section{Milestone Description}\label{milestone-description}

Increased complexity of the simple model to include data from Milestone
5. Automated testing and validating procedures implemented where
possible. Procedures for multiple models comparison and accounting for
uncertainty established.

\section{Summary of Milestone 4}\label{summary-of-milestone-4}

In Milestone 4, we presented a simple transmission model that made us of
historical case counts, tranmissibility of the pathogen and geographical
characetrisation as a sole means to predict future risk. The number of
cases at a location \(j\) at time \(t\) is given by the equation \[
  I_{j, t} \sim Pois\left( \sum_{i = 1}^{n} {\left( p_{i \rightarrow j}
  R_{t, i} \sum_{s = 1}^{t}{I_{i, t - s} w_{s}}\right)} \right),
\]

where \(R_{t, i}\) is the reproduction number at location \(i\) at time
\(t\) and \(p_{i \rightarrow j}\) is the probability of moving from
location \(i\) to location \(j\). The probability of moving between
locations is derived from the relative flow of populations between
locations. This latter quantity is estimated using a population flow
model. In the previous iteration of our work, we had used gravity model.
Under this model, the flow of individuals from area \(i\) to area \(j\),
\(\phi_{i \rightarrow j}\), is proportional to the product of the
populations of the two areas, \(N_i\) and \(N_j\), and inversely
proportional to the distance between them \(d_{i, j}\) raised to some
power. That is,
\[  p_{i \rightarrow j} = (1 - p_{stay}^i) r_{i \rightarrow j}^{spread},\]

where \(p_{stay}^i\) is the probability of staying at location \(i\) and
\(r_{i \rightarrow j}^{spread}\) is the relative risk of spread to a
location \(j\) from a focal location \(i\).
\(r_{i \rightarrow j}^{spread}\) can be quantified as \[
  r_{i \rightarrow j}^{spread} = \frac{\phi_{i \rightarrow
  j}}{\sum_{x}{\phi_{i \rightarrow
  j}}}.
\]

\section{Changes in Milestone 6}\label{changes-in-milestone-6}

\hypertarget{workflow}{\subsection{Workflow}\label{workflow}}

The pre-processing steps for data obrained from HealthMap/ProMed consist
of:
\begin{itemize}
\item Restricting the full data set to the locations of interest;
\item For days with multiple alerts, merge the alerts into a single
  alert;
\item Discard inconsistent data i.e., that which lead to a decreasing
  cumulative counts;
\item Interpolating for dates for which data are missing
so that the cumulative incidence count is available for every day in the
interval of interest and retrieve incidence data from interpolated
cumulative incidence data.
\end{itemize}

In addition to the above, a step was added after merging duplicate
alerts to remove outliers. This was done using Chebyshev Inequality
with sample mean (see~\ref{saw1984chebyshev}). 

\subsection{Model training and validation using data from
WHO}\label{model-training-and-validation-using-data-from-who}

In the current iteration, the model was trained and validated the data
on cases officially reported to the WHO during the 2013--2016 Ebola
outbreak in Guinea, Liberia and Sierra Leone. This dataset was cleaned
and published in \citep{garske20160308} and it is this cleaned version
of the data that were used in this work. This dataset consists of
incidence reports at ADM2 level. Thus in using it, we were able to
better validate the model. We refer to this dataset as WHO data
throughout the rest of this document.

\section{Output for Ebola in West
Africa}\label{output-for-ebola-in-west-africa}

In this section, we provide results from each step in the workflow
described in \protect\hyperlink{workflow}{Section}.

\subsection{Incidence data clean-up}\label{incidence-data-clean-up}

Since the WHO data used was a clean version, pre-processing was not
needed. The pre-processing steps are illustrated using data from
HealthMap. Figure~\ref{fig:lbr_cleanup} shows
the HealthMap data for Liberia through each pre-processing step.

\begin{figure}
  \includegraphics[]{"ms6-figures/liberia-preprocessing"}
  \label{fig:lbr_cleanup}
  \caption{An illustration of the pre-processing carried out on
    HealthMap data. Note the outliers that have removed at the third
    step.}
\end{figure}

\subsection{Estimating
Transmissibility}\label{estimating-transmissibility}

\subsection{Predicting Future Cases}\label{predicting-future-cases}

\subsection{Connectivity and Risk of International
Spread}\label{connectivity-and-risk-of-international-spread}

\section{Conclusions and Next Steps}\label{conclusions-and-next-steps}

\renewcommand\refname{References}
\bibliography{ms6.bib}


\end{document}
