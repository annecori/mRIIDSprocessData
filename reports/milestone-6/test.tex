\documentclass[a4paper,12pt]{article}
\usepackage[utf8]{inputenc}
\usepackage{amsmath,amssymb,amsthm}
\usepackage{natbib}
\usepackage[margin=3cm]{geometry}
\usepackage{todonotes}
\usepackage{mathtools}
\usepackage{subfiles}
\usepackage{subcaption}
\usepackage{booktabs}

\begin{document}

\textbf{The model fits the data well and is able to capture the
  strength and direction of the trend. Table~\ref{tab:fitgoodness}
  shows the percentage of data points that lie within the 95\%
  confidence interval of the predictions. The analysis is computationally
  light and can be carried out in real time. \newline If the direction of the incidence trend is reversed
  (i.e., case numbers start declining after a period of growth or vice
  versa), the model does a poor job since an underlying assumption of
  the model is that things will continue unchanged.}



\begin{center}
  \begin{table}
  \begin{tabular}{ccc}
    District & At 300 days & At 500 days \\
    \hline
    BO &  57.1 & \\
    
    BOMBALI & 14.3 & \\
    
    KAILAHUN & 42.9 & \\
    
    KAMBIA & 28.6 & \\
    
    KENEMA & 42.9 & \\
    
    KOINADUGU & 14.3 & \\
    
    KONO & 85.7 & \\
    
    MOYAMBA & 14.3 & \\
    
    PORTLOKO & 14.3 & \\
    
    PUJEHUN & 71.4 & \\
    
    WESTERN & 28 & \\
\end{tabular}
    \caption{qwerty}
    \label{tab:fitgoodness}
  
  \end{table}
\end{center}





  


\end{document}