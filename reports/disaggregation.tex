\documentclass[11pt,]{article}
\usepackage[]{mathpazo}
\usepackage{amssymb,amsmath}
\usepackage{ifxetex,ifluatex}
\usepackage{fixltx2e} % provides \textsubscript
\ifnum 0\ifxetex 1\fi\ifluatex 1\fi=0 % if pdftex
  \usepackage[T1]{fontenc}
  \usepackage[utf8]{inputenc}
\else % if luatex or xelatex
  \ifxetex
    \usepackage{mathspec}
  \else
    \usepackage{fontspec}
  \fi
  \defaultfontfeatures{Ligatures=TeX,Scale=MatchLowercase}
\fi
% use upquote if available, for straight quotes in verbatim environments
\IfFileExists{upquote.sty}{\usepackage{upquote}}{}
% use microtype if available
\IfFileExists{microtype.sty}{%
\usepackage{microtype}
\UseMicrotypeSet[protrusion]{basicmath} % disable protrusion for tt fonts
}{}
\usepackage[margin=1in]{geometry}
\usepackage{hyperref}
\hypersetup{unicode=true,
            pdftitle={Borrowing information across spatial scales},
            pdfborder={0 0 0},
            breaklinks=true}
\urlstyle{same}  % don't use monospace font for urls
\usepackage{natbib}
\bibliographystyle{apsr}
\usepackage{graphicx,grffile}
\makeatletter
\def\maxwidth{\ifdim\Gin@nat@width>\linewidth\linewidth\else\Gin@nat@width\fi}
\def\maxheight{\ifdim\Gin@nat@height>\textheight\textheight\else\Gin@nat@height\fi}
\makeatother
% Scale images if necessary, so that they will not overflow the page
% margins by default, and it is still possible to overwrite the defaults
% using explicit options in \includegraphics[width, height, ...]{}
\setkeys{Gin}{width=\maxwidth,height=\maxheight,keepaspectratio}
\IfFileExists{parskip.sty}{%
\usepackage{parskip}
}{% else
\setlength{\parindent}{0pt}
\setlength{\parskip}{6pt plus 2pt minus 1pt}
}
\setlength{\emergencystretch}{3em}  % prevent overfull lines
\providecommand{\tightlist}{%
  \setlength{\itemsep}{0pt}\setlength{\parskip}{0pt}}
\setcounter{secnumdepth}{0}
% Redefines (sub)paragraphs to behave more like sections
\ifx\paragraph\undefined\else
\let\oldparagraph\paragraph
\renewcommand{\paragraph}[1]{\oldparagraph{#1}\mbox{}}
\fi
\ifx\subparagraph\undefined\else
\let\oldsubparagraph\subparagraph
\renewcommand{\subparagraph}[1]{\oldsubparagraph{#1}\mbox{}}
\fi

%%% Use protect on footnotes to avoid problems with footnotes in titles
\let\rmarkdownfootnote\footnote%
\def\footnote{\protect\rmarkdownfootnote}

%%% Change title format to be more compact
\usepackage{titling}

% Create subtitle command for use in maketitle
\newcommand{\subtitle}[1]{
  \posttitle{
    \begin{center}\large#1\end{center}
    }
}

\setlength{\droptitle}{-2em}
  \title{Borrowing information across spatial scales}
  \pretitle{\vspace{\droptitle}\centering\huge}
  \posttitle{\par}
  \author{true}
  \preauthor{\centering\large\emph}
  \postauthor{\par}
  \predate{\centering\large\emph}
  \postdate{\par}
  \date{Y}


\begin{document}
\maketitle

\subsection{Unpack parameters}\label{unpack-parameters}

\subsection{Model}\label{model}

We will assign the national case count to districts according to a
multinomial distribution with the probabilities derived from gravity
model or simply population density.

First we read in the data that is required by both methods.

\subsection{Ebola Parameters}\label{ebola-parameters}

\subsection{Incidence data from
HealthMap}\label{incidence-data-from-healthmap}

This is the aggregated data that will be used to assign cases to
districts.

We will work with weekly incidence counts.

Also add isoweek column so that we can compare across the three data
sets.

\subsection{Weekly Training Incidence
Data}\label{weekly-training-incidence-data}

\subsection{Validation data}\label{validation-data}

The performance of the model will be validated against the cleaned-up
data.

\subsection{Further cleaning}\label{further-cleaning}

Maybe get rid of PUJEHUN as it has low incidence (difficult to fit any
model)

\subsection{Method 1: Relative to the reproduction number from recent
past}\label{method-1-relative-to-the-reproduction-number-from-recent-past}

\subsubsection{3A: Use data published by
WHO}\label{a-use-data-published-by-who}

Suppose we are doing the disaggregation in the last week of 2014. Read
in the latest information published by WHO. Start binning at date that
ends week in variable ``bin\_from''. It might be better to use recent
data rather than all data available upto the point of projection.

\subsection{Fitting log-linear model}\label{fitting-log-linear-model}

Add 1 to avoid incidence::fit rejecting all 0s and throwing an error.
Dubious but lets go with it for now.

\subsection{Assign cases to districts}\label{assign-cases-to-districts}

Extract the data we want to use for disaggreation.

Write out the output.

Also write out the training data.

\subsection{Training Data}\label{training-data}

\subsection{Visualizing the results}\label{visualizing-the-results}

\subsection{Validation data}\label{validation-data-1}

\includegraphics{figures/unnamed-chunk-9-1.pdf}


\end{document}
