\documentclass[11pt,]{article}
\usepackage[]{mathpazo}
\usepackage{amssymb,amsmath}
\usepackage{ifxetex,ifluatex}
\usepackage{fixltx2e} % provides \textsubscript
\ifnum 0\ifxetex 1\fi\ifluatex 1\fi=0 % if pdftex
  \usepackage[T1]{fontenc}
  \usepackage[utf8]{inputenc}
\else % if luatex or xelatex
  \ifxetex
    \usepackage{mathspec}
  \else
    \usepackage{fontspec}
  \fi
  \defaultfontfeatures{Ligatures=TeX,Scale=MatchLowercase}
\fi
% use upquote if available, for straight quotes in verbatim environments
\IfFileExists{upquote.sty}{\usepackage{upquote}}{}
% use microtype if available
\IfFileExists{microtype.sty}{%
\usepackage{microtype}
\UseMicrotypeSet[protrusion]{basicmath} % disable protrusion for tt fonts
}{}
\usepackage[margin=1in]{geometry}
\usepackage{hyperref}
\hypersetup{unicode=true,
            pdftitle={WHO Data Pre-processing},
            pdfborder={0 0 0},
            breaklinks=true}
\urlstyle{same}  % don't use monospace font for urls
\usepackage{natbib}
\bibliographystyle{apsr}
\usepackage{graphicx,grffile}
\makeatletter
\def\maxwidth{\ifdim\Gin@nat@width>\linewidth\linewidth\else\Gin@nat@width\fi}
\def\maxheight{\ifdim\Gin@nat@height>\textheight\textheight\else\Gin@nat@height\fi}
\makeatother
% Scale images if necessary, so that they will not overflow the page
% margins by default, and it is still possible to overwrite the defaults
% using explicit options in \includegraphics[width, height, ...]{}
\setkeys{Gin}{width=\maxwidth,height=\maxheight,keepaspectratio}
\IfFileExists{parskip.sty}{%
\usepackage{parskip}
}{% else
\setlength{\parindent}{0pt}
\setlength{\parskip}{6pt plus 2pt minus 1pt}
}
\setlength{\emergencystretch}{3em}  % prevent overfull lines
\providecommand{\tightlist}{%
  \setlength{\itemsep}{0pt}\setlength{\parskip}{0pt}}
\setcounter{secnumdepth}{0}
% Redefines (sub)paragraphs to behave more like sections
\ifx\paragraph\undefined\else
\let\oldparagraph\paragraph
\renewcommand{\paragraph}[1]{\oldparagraph{#1}\mbox{}}
\fi
\ifx\subparagraph\undefined\else
\let\oldsubparagraph\subparagraph
\renewcommand{\subparagraph}[1]{\oldsubparagraph{#1}\mbox{}}
\fi

%%% Use protect on footnotes to avoid problems with footnotes in titles
\let\rmarkdownfootnote\footnote%
\def\footnote{\protect\rmarkdownfootnote}

%%% Change title format to be more compact
\usepackage{titling}

% Create subtitle command for use in maketitle
\newcommand{\subtitle}[1]{
  \posttitle{
    \begin{center}\large#1\end{center}
    }
}

\setlength{\droptitle}{-2em}
  \title{WHO Data Pre-processing}
  \pretitle{\vspace{\droptitle}\centering\huge}
  \posttitle{\par}
  \author{true}
  \preauthor{\centering\large\emph}
  \postauthor{\par}
  \predate{\centering\large\emph}
  \postdate{\par}
  \date{Y}


\begin{document}
\maketitle

\section{Introduction}\label{introduction}

The pre-processing steps consist of turning the line count into a daily
incidence series and turning the given tall data into wide data.

The current data set has three possible values for the epidemiological
case definition - confirmed, probable and suspected. Epidemiological
case definition of probable and suspected cases differed across
countries while confirmed cases were the ones confirmed through lab
report. For the purpose of the current analysis, one approach could be
to lump all of them together. The inferred date of onset (rather than
the date reported) is used for estimation. The columns we use are :
Country, EpiCaseDef (probably), DateOnsetInferred and CL\_DistrictRes.

Some spelling variations between the WHO data and the data set
containing the district co-ordinates need to be taken into account.

\subsection{Grouping by districts and interpolating missing
data}\label{grouping-by-districts-and-interpolating-missing-data}

For each district in a country, add the number of records for each date
to get incidence count.

Within each district, if there is a date on which no cases are recorded,
we assume that there were no cases on that date and add this to the
record. At the end of this step, the incidence time series for each
district should be a daily time series.

\includegraphics{figures/who_raw_viz-1.pdf}
\includegraphics{figures/who_raw_viz-2.pdf}

Write the files.

Now we will convert the data from the tall to the wide format.

\begin{verbatim}
## $Guinea
## NULL
## 
## $Liberia
## NULL
## 
## $`Sierra Leone`
## NULL
\end{verbatim}

\begin{verbatim}
## $Guinea
## # A tibble: 633 x 28
##          Date BEYLA BOFFA  BOKE CONAKRY COYAH DABOLA DALABA DINGUIRAYE
##  *     <date> <dbl> <dbl> <dbl>   <dbl> <dbl>  <dbl>  <dbl>      <dbl>
##  1 2014-01-01     0     0     0       0     0      0      0          0
##  2 2014-01-02     0     0     0       0     0      0      0          0
##  3 2014-01-03     0     0     0       0     0      0      0          0
##  4 2014-01-04     0     0     0       0     0      0      0          0
##  5 2014-01-05     0     0     0       0     0      0      0          0
##  6 2014-01-06     0     0     0       0     0      0      0          0
##  7 2014-01-07     0     0     0       0     0      0      0          0
##  8 2014-01-08     0     0     0       0     0      0      0          0
##  9 2014-01-09     0     0     0       0     0      0      0          0
## 10 2014-01-10     0     0     0       0     0      0      0          0
## # ... with 623 more rows, and 19 more variables: DUBREKA <dbl>,
## #   FARANAH <dbl>, FORECARIAH <dbl>, FRIA <dbl>, GUECKEDOU <dbl>,
## #   KANKAN <dbl>, KEROUANE <dbl>, KINDIA <dbl>, KISSIDOUGOU <dbl>,
## #   KOUROUSSA <dbl>, LOLA <dbl>, MACENTA <dbl>, MALI <dbl>,
## #   NZEREKORE <dbl>, PITA <dbl>, SIGUIRI <dbl>, TELIMELE <dbl>,
## #   TOUGUE <dbl>, YAMOU <dbl>
## 
## $Liberia
## # A tibble: 329 x 16
##          Date  BOMI  BONG GBAPOLU GRANDBASSA GRANDCAPEMOUNT GRANDGEDEH
##  *     <date> <dbl> <dbl>   <dbl>      <dbl>          <dbl>      <dbl>
##  1 2014-02-23     0     0       0          0              0          0
##  2 2014-02-24     0     1       0          0              0          0
##  3 2014-02-25     0     0       0          0              0          0
##  4 2014-02-26     0     0       0          0              0          0
##  5 2014-02-27     0     0       0          0              0          0
##  6 2014-02-28     0     0       0          0              0          0
##  7 2014-03-01     0     0       0          0              0          0
##  8 2014-03-02     0     0       0          0              0          0
##  9 2014-03-03     1     0       0          0              0          0
## 10 2014-03-04     0     0       0          0              0          0
## # ... with 319 more rows, and 9 more variables: GRANDKRU <dbl>,
## #   LOFA <dbl>, MARGIBI <dbl>, MARYLAND <dbl>, MONTSERRADO <dbl>,
## #   NIMBA <dbl>, RIVERCESS <dbl>, RIVERGEE <dbl>, SINOE <dbl>
## 
## $`Sierra Leone`
## # A tibble: 606 x 14
##          Date    BO BOMBALI BONTHE KAILAHUN KAMBIA KENEMA KOINADUGU  KONO
##  *     <date> <dbl>   <dbl>  <dbl>    <dbl>  <dbl>  <dbl>     <dbl> <dbl>
##  1 2013-12-26     0       0      0        0      0      0         0     0
##  2 2013-12-27     0       0      0        0      0      0         0     0
##  3 2013-12-28     0       0      0        0      0      0         0     0
##  4 2013-12-29     0       0      0        0      0      0         0     0
##  5 2013-12-30     0       0      0        0      0      0         0     0
##  6 2013-12-31     0       0      0        0      0      0         0     0
##  7 2014-01-01     0       0      0        0      0      0         0     0
##  8 2014-01-02     0       0      0        0      0      0         0     0
##  9 2014-01-03     0       0      0        0      0      0         0     0
## 10 2014-01-04     0       0      0        0      0      0         0     0
## # ... with 596 more rows, and 5 more variables: MOYAMBA <dbl>,
## #   PORTLOKO <dbl>, PUJEHUN <dbl>, TONKOLILI <dbl>, WESTERN <dbl>
\end{verbatim}

Also write out the data for all districts for later analysis.


\end{document}
